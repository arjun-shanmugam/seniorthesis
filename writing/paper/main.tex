\documentclass[12pt]{article}

\usepackage{amssymb,amsmath,amsfonts, booktabs, dsfont, eurosym,float,geometry,ulem,graphicx,caption,color,setspace,sectsty,comment,footmisc,caption,multicol, multirow, natbib, pdflscape,array,hyperref}
\bibliographystyle{rusnat}


\normalem

\geometry{left=0.5in,right=0.5in,top=1.0in,bottom=1.0in}

\begin{document}



\begin{titlepage}
\title{Arjun Shanmugam's Senior Thesis}
\author{Arjun Shanmugam}
\date{\today}
\maketitle
\begin{abstract}
\noindent Placeholder\\

\bigskip
\end{abstract}
\setcounter{page}{0}
\thispagestyle{empty}
\end{titlepage}
\pagebreak \newpage

\doublespacing

\section{Introduction} \label{sec:introduction}
\bibliography{writing/paper/citations}
\subsection{Literature Review}

\section{Institutional Context}
    \subsection{Eviction in Massachusetts}
        \subsubsection{The Massachusetts Housing Court}
        \subsubsection{The Eviction Process}
    \subsection{Property Tax Assessment}
        \subsubsection{The Property Value Assessment Process}
    \subsection{Zestimates}
        \subsubsection{How Are Zestimates Produced?}
        \subsubsection{Reliability}
\section{Data} \label{sec:data}
    \begin{landscape}
    \subsection{Evictions Data}
        \begin{figure}[H]
            \centering
            \includegraphics{output/summary_statistics/figures/evictions_map.png}
            \caption{Spatial Incidence of Eviction}
            \label{fig:my_label}
        \end{figure}

        \begin{figure}[H]
            \centering
            \includegraphics{output/summary_statistics/figures/filings_over_time.png}
            \caption{Eviction Filings Over Time}
            \label{fig:my_label}
        \end{figure}
    \end{landscape}
        \begin{table}[H]
            \centering
            \begin{tabular}{lccc}
\toprule
 & Cases Won By Defendant & Cases Won By Plaintiff & Portion of All Cases \\
Last Docket Date &  &  &  \\
\midrule
All Months & 620 & 978 & 1.00 \\
2019-06 & 43 & 19 & 0.04 \\
2019-07 & 60 & 50 & 0.07 \\
2019-08 & 54 & 117 & 0.11 \\
2019-09 & 52 & 139 & 0.12 \\
2019-10 & 80 & 108 & 0.12 \\
2019-11 & 59 & 72 & 0.08 \\
2019-12 & 59 & 60 & 0.07 \\
2020-01 & 65 & 128 & 0.12 \\
2020-02 & 74 & 125 & 0.12 \\
2020-03 & 74 & 160 & 0.15 \\
\bottomrule
\end{tabular}

            \caption{Distribution of Eviction Filings and Outcomes}
            \label{tab:my_label}
        \end{table}

        \begin{figure}[H]
            \centering
            \includegraphics[scale=0.8]{output/DiD/figures/trends_in_zestimates_by_cohort.png}
            \caption{Trends in Zestimates by Cohort}
            \label{fig:my_label}
        \end{figure}

    \subsection{Summary Statistics}
         \begin{table}[H]
            \centering
            \small
            \begin{tabular}{llcccc}
\toprule
 &  & Mean & Median & S.D. & N \\
Panel & Variable &  &  &  &  \\
\midrule
\multirow[c]{14}{4cm}{\textit{Panel A: Pre-treatment Outcomes}} & Zestimate, Jan. 2017 & 298,898.22 & 235,078.00 & 352,756.46 & 10,176 \\
 & pre_treatment_change_in_any_crime_140m & 0.02 & 0.00 & 2.42 & 40,734 \\
 & pre_treatment_change_in_any_crime_200m & 0.05 & 0.00 & 3.31 & 40,734 \\
 & pre_treatment_change_in_any_crime_60m & 0.01 & 0.00 & 1.16 & 40,734 \\
 & pre_treatment_change_in_any_crime_60m & 0.01 & 0.00 & 1.16 & 40,734 \\
 & pre_treatment_change_in_any_crime_90m & 0.03 & 0.00 & 1.71 & 40,734 \\
 & pre_treatment_change_in_any_crime_90m & 0.03 & 0.00 & 1.71 & 40,734 \\
 & pre_treatment_change_in_zestimate & 46,593.08 & 37,330.00 & 158,125.24 & 10,160 \\
 & twenty_seventeen_any_crime_140m & 1.41 & 0.00 & 5.01 & 40,734 \\
 & twenty_seventeen_any_crime_200m & 2.68 & 0.00 & 8.69 & 40,734 \\
 & twenty_seventeen_any_crime_60m & 0.29 & 0.00 & 1.31 & 40,734 \\
 & twenty_seventeen_any_crime_60m & 0.29 & 0.00 & 1.31 & 40,734 \\
 & twenty_seventeen_any_crime_90m & 0.62 & 0.00 & 2.42 & 40,734 \\
 & twenty_seventeen_any_crime_90m & 0.62 & 0.00 & 2.42 & 40,734 \\
\cline{1-6}
\multirow[c]{8}{4cm}{\textit{Panel B: Census Tract Characteristics}} & Jobs per square mile (2010) & 4,802.71 & 1,355.40 & 17,583.20 & 40,732 \\
 & Median household income (2016) & 52,695.02 & 47,105.00 & 27,212.35 & 40,732 \\
 & Median two bedroom rent (2015) & 1,116.73 & 1,055.00 & 396.69 & 30,546 \\
 & Population density (2010) & 9,151.92 & 5,982.68 & 9,537.28 & 40,732 \\
 & Share below poverty line & 0.20 & 0.16 & 0.15 & 40,732 \\
 & Share white (2010) & 0.58 & 0.63 & 0.29 & 40,732 \\
 & Share with bachelor's degree & 0.26 & 0.22 & 0.17 & 40,732 \\
 & Share with commute $<$15 minutes (2010) & 0.28 & 0.27 & 0.11 & 40,732 \\
\cline{1-6}
\multirow[c]{3}{4cm}{\textit{Panel C: Case Initiation}} & For cause & 0.12 & 0.00 & 0.33 & 40,734 \\
 & No cause & 0.11 & 0.00 & 0.31 & 40,734 \\
 & Non-payment of rent & 0.75 & 1.00 & 0.43 & 40,734 \\
\cline{1-6}
\multirow[c]{4}{4cm}{\textit{Panel D: Defendant and Plaintiff Characteristics}} & Defendant has an attorney & 0.09 & 0.00 & 0.28 & 40,734 \\
 & Defendant is an entity & 0.01 & 0.00 & 0.08 & 40,734 \\
 & Plaintiff has an attorney & 0.84 & 1.00 & 0.37 & 40,734 \\
 & Plaintiff is an entity & 0.70 & 1.00 & 0.46 & 40,734 \\
\cline{1-6}
\multirow[c]{6}{4cm}{\textit{Panel E: Case Resolution}} & Case defaulted & 0.20 & 0.00 & 0.40 & 40,734 \\
 & Case dismised & 0.29 & 0.00 & 0.45 & 40,734 \\
 & Case duration & 57.81 & 21.00 & 78.28 & 39,094 \\
 & Case heard & 0.06 & 0.00 & 0.23 & 40,734 \\
 & Case mediated & 0.41 & 0.00 & 0.49 & 40,734 \\
 & Money judgment & 1,890.02 & 0.00 & 5,279.08 & 40,734 \\
\cline{1-6}
\multirow[c]{14}{4cm}{\textit{Panel F: Post-treatment Outcomes}} & Zestimate one year after filing date & 421,590.21 & 347,444.00 & 344,333.24 & 10,443 \\
 & Zestimate two years after filing date & 477,931.84 & 393,800.00 & 457,699.06 & 9,175 \\
 & any_crime_140m_1_years_relative_to_treatment & 1.41 & 0.00 & 4.90 & 40,734 \\
 & any_crime_140m_2_years_relative_to_treatment & 1.50 & 0.00 & 5.04 & 35,684 \\
 & any_crime_200m_1_years_relative_to_treatment & 2.65 & 0.00 & 8.37 & 40,734 \\
 & any_crime_200m_2_years_relative_to_treatment & 2.84 & 0.00 & 8.74 & 35,684 \\
 & any_crime_60m_1_years_relative_to_treatment & 0.31 & 0.00 & 1.41 & 40,734 \\
 & any_crime_60m_1_years_relative_to_treatment & 0.31 & 0.00 & 1.41 & 40,734 \\
 & any_crime_60m_2_years_relative_to_treatment & 0.34 & 0.00 & 1.44 & 35,684 \\
 & any_crime_60m_2_years_relative_to_treatment & 0.34 & 0.00 & 1.44 & 35,684 \\
 & any_crime_90m_1_years_relative_to_treatment & 0.66 & 0.00 & 2.50 & 40,734 \\
 & any_crime_90m_1_years_relative_to_treatment & 0.66 & 0.00 & 2.50 & 40,734 \\
 & any_crime_90m_2_years_relative_to_treatment & 0.69 & 0.00 & 2.53 & 35,684 \\
 & any_crime_90m_2_years_relative_to_treatment & 0.69 & 0.00 & 2.53 & 35,684 \\
\cline{1-6}
\bottomrule
\end{tabular}

            \caption{Summary Statistics}
            \label{tab:table_1}
        \end{table}
        \newpage

\section{Empirical Strategy: Difference-in-Differences}
    I seek to estimate the average treatment effects of plaintiff victory in eviction cases on treated properties. I use the staggered difference-in-difference estimator proposed in \cite{callaway_difference--differences_2021}, which uses two-period, two-unit difference-in-difference estimators to estimate time- and cohort-specific ATTs and then aggregates them, weighting by cohort size, to produce summaries of the ATT. 
    
    
    Let $G_i$ be the month during which the eviction case involving property $i$ was filed, such that $G_i = g \in \{\text{May} \; 2019, \text{June} \; 2019, ..., \text{May} \; 2021\}$. Let $C_i = 1$ if the eviction case involving $i$ resulted in a victory for the defendant and $0$ otherwise. If $G_i = g$ and $C_i = 0$, then property $i$ is a treated property and a member of the cohort first treated during month $g$ (cohort $g$). If $C_i=1$, then property $i$ is a never-treated property. Let $Y_{i,t}$ be property $i$'s Zestimate during month $t$ and define $\Delta Y_{i, g-1, t} \equiv Y_{i,t} - Y_{i,g-1}$ so that $\Delta Y_{i, g-1, t}$ is the change in property $i$'s Zestimate between months $t$ and $g-1$.

    \subsection{Unconditional Estimates of the ATT}
    The following is an unconditional estimator for $ATT(g,t)$, the average treatment affect during month $t$ for cohort $g$.
    \begin{align}
        \hat{ATT}^{nev}_{un}(g, t) = \frac{\sum_i\Delta Y_{i, g-1, t}\mathds{1}\{G_i=g\}}{\sum_i\mathds{1}\{G_i=g\}} - \frac{\sum_i\Delta Y_{i, g-1, t}\mathds{1}\{C_i=1\}}{\sum_i\mathds{1}\{C_i=1\}}
    \end{align}
    The above estimator will identify $ATT(g,t)$ under the assumption that the change in untreated potential outcomes between periods $g-1$ and $t$ is the same among units in cohort $g$ as it is among never-treated units.

    \subsection{D.R. Estimates of the ATT}

    The above unconditional parallel trends assumption is unlikely to hold, as eviction case outcomes and zestimates may be related to their socioeconomic surroundings. Table 3 explores this theory further, reporting results from univariate regressions of January 2022 Zestimates and an indicator for plaintiff victory on the pre-treatment characteristics listed in Panels A through D of Table 2. Each cell gives the p-value from a regression of the variable which labels its column on the variable which labels its row. Table 3 shows significant associations between pre-treatment characteristics and Zestimates and pre-treatment characteristics and case outcomes, suggesting that the unconditional parallel trends assumption imposed earlier may not be valid.

    \begin{table}[H]
        \centering
        \begin{tabular}{lcc}
\toprule
 &  & \multicolumn{2}{r}{Dependent Variable} \\
 &  & Zestimate, Dec. 2022 & Plaintiff victory \\
\midrule
\multirow[c]{3}{*}{Panel A: Pre-treatment Zestimates} & Zestimate, Jan. 2017 & 0.00 & 0.00 \\
 & Zestimate, Jan. 2018 & 0.00 & 0.00 \\
 & Change from Jan. 2018 to Jan. 2019 & 0.00 & 0.27 \\
\multirow[c]{3}{*}{Panel B: Census Tract Characteristics} & Median household income (2016) & 0.00 & 0.01 \\
 & Population density (2010) & 0.00 & 0.01 \\
 & Portion white (2010) & 0.00 & 0.00 \\
\multirow[c]{3}{*}{Panel C: Case Initiation} & For cause & 0.01 & 0.68 \\
 & No cause & 0.98 & 0.03 \\
 & Non-payment of rent & 0.31 & 0.76 \\
\multirow[c]{4}{*}{Panel D: Defendant and Plaintiff Characteristics} & Defendant has an attorney & 0.37 & 0.00 \\
 & Plaintiff has an attorney & 0.61 & 0.48 \\
 & Defendant is an entity & 0.02 & 0.37 \\
 & Plaintiff is an entity & 0.63 & 0.12 \\
\multirow[c]{6}{*}{Panel E: Case Resolution} & Case duration & 0.57 & 0.00 \\
 & Case defaulted & 0.89 & 0.00 \\
 & Case dismised & 0.00 & 0.00 \\
 & Case heard & 0.25 & 0.00 \\
 & Money judgment & 0.88 & 0.00 \\
 & Case mediated & nan & nan \\
\multirow[c]{4}{*}{Panel F: Post-treatment Zestimates} & At filing date & 0.00 & 0.04 \\
 & One year after filing date & 0.00 & 0.05 \\
 & Two years after filing date & 0.00 & 0.01 \\
 & Three years after filing date & 0.00 & 0.08 \\
\bottomrule
\end{tabular}

        \caption{Relationship Between Pre-Treatment Characteristics and the Dependent and Independent Variables}
        \label{tab:my_label}
    \end{table}

    My next strategy uses covariates to construct a more credible counterfactual for the observed path of outcomes in the treatment group using the doubly robust difference-in-differences estimator proposed by \cite{santanna_doubly_2018}. For each property, let $X_i$ be a vector containing the covariates whose univariate regression coefficients from column 1 of table 3 are significant at the 5 percent level. To attempt to identify $ATT(g, t)$ using the doubly robust estimator, I first estimate $\hat{p}_g(X_i)$, a logit regression propensity score model for being in cohort $g$. I assign a weight $\hat{w}_i(X_i) \equiv\frac{\hat{p}_g(X_i)}{1 - \hat{p}_g(X_i)}$ to each never-treated property $i$; $\hat{w}_i(X_i) \equiv 1$ for treated properties. Define $\hat{w}^*_i(X_i) = \frac{\hat{w}_i(X_i)}{\sum_i \hat{w}_i(X_i)}$. Second, using only never-treated counties, I regress $\Delta Y_{i, g-1, t}$ on $X_j$, weighting by $\hat{w}_i(X_i)$. Using the estimated coefficients $\hat{\beta}_{g-1, t}^{X}$, I define $\Delta \hat{\mu}_{g-1, t}(X_i) \equiv \hat{\beta}_{g-1, t}^{X}X_i$ so that $\Delta \hat{\mu}_{g-1, t}(X_i)$ is the predicted change in property $i$'s Zestimate between months $t$ and $g-1$.
    The doubly robust estimator for $ATT(g, t)$ is as follows.
    \begin{align}
        \hat{ATT}^{nev}_{DR, X}(g,t) = \frac{1}{N}\sum_i[(\frac{D_i}{\Bar{D_i}} - \frac{\hat{w}^*_i(X_i)C_i}{\Bar{C_i}})(\Delta Y_{i, g-1, t} - \hat{\mu}_{g-1, t}(X_i))]
    \end{align}
    Note that $\Bar{D_i}$ and $\Bar{C_i}$ are sample averages. Under the assumption of parallel trends among units with the same covariates, $\hat{ATT}^{nev}_{DR, X}(g,t)$ will identify $ATT(g, t)$.
    

    Columns 2 and 3 of table 4 shows significant pre-treatment imbalance in covariates between the treatment and control groups. Each row of column 2 reports the coefficient from a univariate regression of one covariate on an indicator for plaintiff victory. Column 3 reports the p-values associated with each of these coefficients. In each row of column 4, I regress one covariate on an indicator for plaintiff victory and $\hat{w}_i(X_i)$ and report the coefficient on the indicator for plaintiff victory. Column 5 reports the p-values associated with each coefficient and shows that conditioning on pre-treatment covariates makes virtually all pre-treatment differences in covariates insignificant.

    The estimator defined in equation 2 simultaneously models the counterfactual change in \emph{observed} outcomes for untreated properties ($\hat{w}_i(X_i)$) and the \emph{predicted} change in outcomes for untreated properties ($\hat{\mu}_{g-1, t}(X_i)$). As long as one of these two models is correctly specified, the doubly robust estimator will identify $ATT(g, t)$.

    \newpage
    \begin{landscape}
       \begin{table}[H]
        \centering
        \begin{tabular}{llccccc}
\toprule
 &  & \textit{} & \multicolumn{4}{c}{\textit{Difference in Cases Won by Defendant}} \\ 
 &  & Cases Won by Plaintiff & Unweighted & \emph{p} & Weighted & \emph{p} \\
\midrule
\multirow[c]{3}{3cm}{\textit{Panel A}} & Zestimate, Jan. 2017 & 291,945.81 & 26,570.00 & 0.00 & 10,202.27 & 0.18 \\
 & Zestimate, Jan. 2018 & 320,176.82 & 27,148.19 & 0.00 & 9,364.98 & 0.26 \\
 & Change from Jan. 2018 to Jan. 2019 & 50,501.64 & 2,664.69 & 0.27 & -133.67 & 0.96 \\
\cline{1-7}
\multirow[c]{6}{3cm}{\textit{Panel B}} & Portion with bachelor's degree or higher (2010) & 0.25 & 0.01 & 0.07 & 0.01 & 0.39 \\
 & Number of jobs per sq. mile (2010) & 4,158.02 & 263.90 & 0.63 & 133.63 & 0.81 \\
 & Median household income (2016) & 52,087.32 & 2,566.82 & 0.01 & 1,178.12 & 0.25 \\
 & Population density (2010) & 9,031.64 & 944.99 & 0.01 & 483.30 & 0.18 \\
 & Portion white (2010) & 0.60 & 0.04 & 0.00 & 0.02 & 0.07 \\
 & Share with commute under 15 min. (2010) & 0.29 & -0.01 & 0.02 & -0.00 & 0.27 \\
\cline{1-7}
\textit{Panel C} & For cause & 0.12 & 0.01 & 0.68 & 0.00 & 0.82 \\
\cline{1-7}
\textit{Panel D} & Defendant is an entity & 0.01 & 0.00 & 0.37 & 0.00 & 0.67 \\
\cline{1-7}
\bottomrule
\end{tabular}

        \caption{Balance Tests}
        \label{tab:my_label}
    \end{table} 
    \end{landscape}

    \subsection{Aggregating Treatment Effects Across Cohort and Time}
        \subsubsection{ATTs Across Cohorts}
        \subsubsection{ATTs Across Time Periods}
        \subsubsection{Event-Study ATTs}
    
\section{Results} \label{sec:result}

    \subsection{Unconditional Estimates of the ATT}

    \begin{figure}[H]
        \centering
        \includegraphics{output/DiD/figures/att_gt_unconditional_event_study_short_horizon.png}
        \caption{}
        \label{fig:my_label}
    \end{figure}

    \begin{figure}[H]
        \centering
        \includegraphics{output/DiD/figures/att_gt_unconditional_event_study_long_horizon.png}
        \caption{}
        \label{fig:my_label}
    \end{figure}
    
    \begin{figure}[H]
        \centering
        \includegraphics{output/DiD/figures/att_gt_unconditional_time.png}
        \caption{}
        \label{fig:my_label}
    \end{figure}

    \subsection{D.R. Estimates of the ATT}


    
    
    \begin{figure}[H]
        \centering
        \includegraphics{output/DiD/figures/att_gt_dr_event_study_short_horizon.png}
        \caption{}
        \label{fig:my_label}
    \end{figure}
    
    \begin{figure}[H]
        \centering
        \includegraphics{output/DiD/figures/att_gt_dr_event_study_long_horizon.png}
        \caption{}
        \label{fig:my_label}
    \end{figure}

    \begin{figure}[H]
        \centering
        \includegraphics{output/DiD/figures/att_gt_dr_time.png}
        \caption{}
        \label{fig:my_label}
    \end{figure}


\section{Conclusion} \label{sec:conclusion}


\bibliography{citations}


\clearpage

\onehalfspacing

%
\end{document}