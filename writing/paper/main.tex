\documentclass[12pt]{article}

\usepackage{amssymb,amsmath,amsfonts,booktabs,eurosym,float,geometry,ulem,graphicx,caption,color,setspace,sectsty,comment,footmisc,caption,multicol, multirow,natbib,pdflscape,subfigure,array,hyperref, pdflscape}

\normalem

\onehalfspacing
\newtheorem{theorem}{Theorem}
\newtheorem{corollary}[theorem]{Corollary}
\newtheorem{proposition}{Proposition}
\newenvironment{proof}[1][Proof]{\noindent\textbf{#1.} }{\ \rule{0.5em}{0.5em}}

\newtheorem{hyp}{Hypothesis}
\newtheorem{subhyp}{Hypothesis}[hyp]
\renewcommand{\thesubhyp}{\thehyp\alph{subhyp}}

\newcommand{\red}[1]{{\color{red} #1}}
\newcommand{\blue}[1]{{\color{blue} #1}}

\newcolumntype{L}[1]{>{\raggedright\let\newline\\arraybackslash\hspace{0pt}}m{#1}}
\newcolumntype{C}[1]{>{\centering\let\newline\\arraybackslash\hspace{0pt}}m{#1}}
\newcolumntype{R}[1]{>{\raggedleft\let\newline\\arraybackslash\hspace{0pt}}m{#1}}

\geometry{left=1.0in,right=1.0in,top=1.0in,bottom=1.0in}

\begin{document}

\begin{titlepage}
\title{Arjun Shanmugam's Senior Thesis}
\author{Arjun Shanmugam}
\date{\today}
\maketitle
\begin{abstract}
\noindent Placeholder\\


\bigskip
\end{abstract}
\setcounter{page}{0}
\thispagestyle{empty}
\end{titlepage}
\pagebreak \newpage




\doublespacing


\section{Introduction} \label{sec:introduction}
\subsection{Literature Review}

\section{Institutional Context}
    \subsection{Eviction in Massachusetts}
        \subsubsection{The Massachusetts Housing Court}
        \subsubsection{The Eviction Process}
    \subsection{Property Tax Assessment}
        \subsubsection{The Property Value Assessment Process}
    \subsection{Zestimates}
        \subsubsection{How Are Zestimates Produced?}
        \subsubsection{Reliability}
\section{Data} \label{sec:data}
    \begin{landscape}
    \subsection{Evictions Data}
        \begin{figure}[H]
            \centering
            \includegraphics{output/summary_statistics/figures/evictions_map.png}
            \caption{Spatial Incidence of Eviction}
            \label{fig:my_label}
        \end{figure}

        \begin{figure}[H]
            \centering
            \includegraphics{output/summary_statistics/figures/filings_over_time.png}
            \caption{Eviction Filings Over Time}
            \label{fig:my_label}
        \end{figure}
    \end{landscape}
        \begin{table}[H]
            \centering
            \begin{tabular}{lccc}
\toprule
 & Cases Won By Defendant & Cases Won By Plaintiff & Portion of All Cases \\
Last Docket Date &  &  &  \\
\midrule
All Months & 620 & 978 & 1.00 \\
2019-06 & 43 & 19 & 0.04 \\
2019-07 & 60 & 50 & 0.07 \\
2019-08 & 54 & 117 & 0.11 \\
2019-09 & 52 & 139 & 0.12 \\
2019-10 & 80 & 108 & 0.12 \\
2019-11 & 59 & 72 & 0.08 \\
2019-12 & 59 & 60 & 0.07 \\
2020-01 & 65 & 128 & 0.12 \\
2020-02 & 74 & 125 & 0.12 \\
2020-03 & 74 & 160 & 0.15 \\
\bottomrule
\end{tabular}

            \caption{Distribution of Eviction Filings and Outcomes}
            \label{tab:my_label}
        \end{table}
    
    \subsection{Tax Assessment Records}
    \subsection{Zestimates}
        \begin{figure}[H]
            \centering
            \includegraphics{output/summary_statistics/figures/trends_in_zestimates.png}
            \caption{Trends in Zestimates}
            \label{fig:my_label}
        \end{figure}
    \begin{landscape}
    \subsection{Summary Statistics}
         \begin{table}[H]
            \centering
            \small
            \begin{tabular}{llcccc}
\toprule
 &  & Mean & Median & S.D. & N \\
Panel & Variable &  &  &  &  \\
\midrule
\multirow[c]{14}{4cm}{\textit{Panel A: Pre-treatment Outcomes}} & Zestimate, Jan. 2017 & 298,898.22 & 235,078.00 & 352,756.46 & 10,176 \\
 & pre_treatment_change_in_any_crime_140m & 0.02 & 0.00 & 2.42 & 40,734 \\
 & pre_treatment_change_in_any_crime_200m & 0.05 & 0.00 & 3.31 & 40,734 \\
 & pre_treatment_change_in_any_crime_60m & 0.01 & 0.00 & 1.16 & 40,734 \\
 & pre_treatment_change_in_any_crime_60m & 0.01 & 0.00 & 1.16 & 40,734 \\
 & pre_treatment_change_in_any_crime_90m & 0.03 & 0.00 & 1.71 & 40,734 \\
 & pre_treatment_change_in_any_crime_90m & 0.03 & 0.00 & 1.71 & 40,734 \\
 & pre_treatment_change_in_zestimate & 46,593.08 & 37,330.00 & 158,125.24 & 10,160 \\
 & twenty_seventeen_any_crime_140m & 1.41 & 0.00 & 5.01 & 40,734 \\
 & twenty_seventeen_any_crime_200m & 2.68 & 0.00 & 8.69 & 40,734 \\
 & twenty_seventeen_any_crime_60m & 0.29 & 0.00 & 1.31 & 40,734 \\
 & twenty_seventeen_any_crime_60m & 0.29 & 0.00 & 1.31 & 40,734 \\
 & twenty_seventeen_any_crime_90m & 0.62 & 0.00 & 2.42 & 40,734 \\
 & twenty_seventeen_any_crime_90m & 0.62 & 0.00 & 2.42 & 40,734 \\
\cline{1-6}
\multirow[c]{8}{4cm}{\textit{Panel B: Census Tract Characteristics}} & Jobs per square mile (2010) & 4,802.71 & 1,355.40 & 17,583.20 & 40,732 \\
 & Median household income (2016) & 52,695.02 & 47,105.00 & 27,212.35 & 40,732 \\
 & Median two bedroom rent (2015) & 1,116.73 & 1,055.00 & 396.69 & 30,546 \\
 & Population density (2010) & 9,151.92 & 5,982.68 & 9,537.28 & 40,732 \\
 & Share below poverty line & 0.20 & 0.16 & 0.15 & 40,732 \\
 & Share white (2010) & 0.58 & 0.63 & 0.29 & 40,732 \\
 & Share with bachelor's degree & 0.26 & 0.22 & 0.17 & 40,732 \\
 & Share with commute $<$15 minutes (2010) & 0.28 & 0.27 & 0.11 & 40,732 \\
\cline{1-6}
\multirow[c]{3}{4cm}{\textit{Panel C: Case Initiation}} & For cause & 0.12 & 0.00 & 0.33 & 40,734 \\
 & No cause & 0.11 & 0.00 & 0.31 & 40,734 \\
 & Non-payment of rent & 0.75 & 1.00 & 0.43 & 40,734 \\
\cline{1-6}
\multirow[c]{4}{4cm}{\textit{Panel D: Defendant and Plaintiff Characteristics}} & Defendant has an attorney & 0.09 & 0.00 & 0.28 & 40,734 \\
 & Defendant is an entity & 0.01 & 0.00 & 0.08 & 40,734 \\
 & Plaintiff has an attorney & 0.84 & 1.00 & 0.37 & 40,734 \\
 & Plaintiff is an entity & 0.70 & 1.00 & 0.46 & 40,734 \\
\cline{1-6}
\multirow[c]{6}{4cm}{\textit{Panel E: Case Resolution}} & Case defaulted & 0.20 & 0.00 & 0.40 & 40,734 \\
 & Case dismised & 0.29 & 0.00 & 0.45 & 40,734 \\
 & Case duration & 57.81 & 21.00 & 78.28 & 39,094 \\
 & Case heard & 0.06 & 0.00 & 0.23 & 40,734 \\
 & Case mediated & 0.41 & 0.00 & 0.49 & 40,734 \\
 & Money judgment & 1,890.02 & 0.00 & 5,279.08 & 40,734 \\
\cline{1-6}
\multirow[c]{14}{4cm}{\textit{Panel F: Post-treatment Outcomes}} & Zestimate one year after filing date & 421,590.21 & 347,444.00 & 344,333.24 & 10,443 \\
 & Zestimate two years after filing date & 477,931.84 & 393,800.00 & 457,699.06 & 9,175 \\
 & any_crime_140m_1_years_relative_to_treatment & 1.41 & 0.00 & 4.90 & 40,734 \\
 & any_crime_140m_2_years_relative_to_treatment & 1.50 & 0.00 & 5.04 & 35,684 \\
 & any_crime_200m_1_years_relative_to_treatment & 2.65 & 0.00 & 8.37 & 40,734 \\
 & any_crime_200m_2_years_relative_to_treatment & 2.84 & 0.00 & 8.74 & 35,684 \\
 & any_crime_60m_1_years_relative_to_treatment & 0.31 & 0.00 & 1.41 & 40,734 \\
 & any_crime_60m_1_years_relative_to_treatment & 0.31 & 0.00 & 1.41 & 40,734 \\
 & any_crime_60m_2_years_relative_to_treatment & 0.34 & 0.00 & 1.44 & 35,684 \\
 & any_crime_60m_2_years_relative_to_treatment & 0.34 & 0.00 & 1.44 & 35,684 \\
 & any_crime_90m_1_years_relative_to_treatment & 0.66 & 0.00 & 2.50 & 40,734 \\
 & any_crime_90m_1_years_relative_to_treatment & 0.66 & 0.00 & 2.50 & 40,734 \\
 & any_crime_90m_2_years_relative_to_treatment & 0.69 & 0.00 & 2.53 & 35,684 \\
 & any_crime_90m_2_years_relative_to_treatment & 0.69 & 0.00 & 2.53 & 35,684 \\
\cline{1-6}
\bottomrule
\end{tabular}

            \caption{Summary Statistics}
            \label{tab:table_1}
        \end{table}
        \newpage
        \begin{table}[H]
            \centering
            \small
            \begin{tabular}{llcccc}
\toprule
 &  & \multicolumn{3}{c}{\textit{Cases Won by Defendant}} & \multicolumn{3}{c}{\textit{Cases Won by Plaintiff}} \\
 &  & Mean & median & N & Mean & median & N \\
Panel & Variable &  &  &  &  &  &  \\
\midrule
\multirow[c]{8}{4cm}{\textit{Panel A: Case Initiation}} & For cause & 0.12 & 0.000000 & 4082 & 0.12 & 0.000000 & 11912 \\
 & For cause (transfer) & 0.01 & 0.000000 & 4082 & 0.01 & 0.000000 & 11912 \\
 & Foreclosure & 0.01 & 0.000000 & 4082 & 0.03 & 0.000000 & 11912 \\
 & Foreclosure (transfer) & 0.00 & 0.000000 & 4082 & 0.00 & 0.000000 & 11912 \\
 & No cause & 0.12 & 0.000000 & 4082 & 0.10 & 0.000000 & 11912 \\
 & No cause (transfer) & 0.01 & 0.000000 & 4082 & 0.00 & 0.000000 & 11912 \\
 & Non-payment of rent & 0.69 & 1.000000 & 4082 & 0.72 & 1.000000 & 11912 \\
 & Non-payment of rent (transfer) & 0.03 & 0.000000 & 4082 & 0.01 & 0.000000 & 11912 \\
\cline{1-8}
\multirow[c]{5}{4cm}{\textit{Panel B: Case Resolution}} & Case duration & 97.63 & 61.000000 & 4082 & 58.85 & 20.000000 & 10273 \\
 & Case defaulted & 0.00 & 0.000000 & 4082 & 0.67 & 1.000000 & 11912 \\
 & Case involuntarily dismissed & 0.99 & 1.000000 & 4082 & 0.00 & 0.000000 & 11912 \\
 & Case heard & 0.01 & 0.000000 & 4082 & 0.19 & 0.000000 & 11912 \\
 & Case mediated & 0.00 & 0.000000 & 4082 & 0.00 & 0.000000 & 11912 \\
\cline{1-8}
\multirow[c]{5}{4cm}{\textit{Panel C: Defendant and Plaintiff Characteristics}} & Defendant has an attorney & 0.20 & 0.000000 & 4082 & 0.07 & 0.000000 & 11912 \\
 & Plaintiff has an attorney & 0.79 & 1.000000 & 4082 & 0.81 & 1.000000 & 11912 \\
 & Defendant is an entity & 0.01 & 0.000000 & 4082 & 0.01 & 0.000000 & 11912 \\
 & Plaintiff is an entity & 0.63 & 1.000000 & 4082 & 0.67 & 1.000000 & 11912 \\
 & Money judgment & 57.67 & 0.000000 & 4082 & 3440.73 & 1,310.940000 & 11912 \\
\cline{1-8}
\multirow[c]{5}{4cm}{\textit{Panel D: Assessor Records From Post-Filing F.Y.}} & Total property value & 6478531.48 & 641,200.000000 & 1256 & 8997593.33 & 579,400.000000 & 3991 \\
 & Building value & 1605427.57 & 169,600.000000 & 1256 & 2352903.60 & 161,000.000000 & 3991 \\
 & Land value & 61247.49 & 1,100.000000 & 1256 & 101018.42 & 2,100.000000 & 3991 \\
 & Other value & 8138231.73 & 823,450.000000 & 1256 & 11440367.23 & 808,100.000000 & 3991 \\
 & Units & 41.23 & 2.000000 & 1256 & 42.10 & 1.000000 & 3991 \\
\cline{1-8}
\multirow[c]{9}{4cm}{\textit{Panel E: Zestimates Around Last Docket Date}} & Five years before latest docket date & 255719.80 & 185,330.500000 & 1018 & 267442.86 & 218,825.000000 & 2901 \\
 & Four years before latest docket date & 278556.70 & 195,892.500000 & 1034 & 285983.57 & 228,526.000000 & 2936 \\
 & Three years before latest docket date & 297696.16 & 210,030.000000 & 1051 & 312711.10 & 244,821.500000 & 3012 \\
 & Two years before latest docket date & 332887.98 & 223,501.000000 & 1054 & 339939.67 & 264,659.000000 & 3013 \\
 & One year before latest docket date & 338734.41 & 240,000.000000 & 1057 & 359532.30 & 285,032.000000 & 3015 \\
 & Latest docket date & 373978.71 & 270,184.000000 & 1043 & 385699.20 & 313,104.500000 & 2990 \\
 & One year after latest docket date & 419565.10 & 317,500.000000 & 1085 & 431037.20 & 354,320.000000 & 3084 \\
 & Two years after latest docket date & 446321.14 & 348,300.000000 & 823 & 490315.74 & 404,600.000000 & 2723 \\
 & Three years after latest docket date & 482787.35 & 363,650.000000 & 454 & 507050.53 & 419,600.000000 & 1791 \\
\cline{1-8}
\bottomrule
\end{tabular}

            \caption{Balance Table}
            \label{tab:my_label}
        \end{table}
    \end{landscape}
    

\section{Empirical Strategy: Difference-in-Difference}
    \subsection{Estimating the ATT in the Staggered Treatment Case}
    I will use Callaway and Sant'Anna' (2021)'s multiperiod, staggered treatment difference-in-difference estimators. The following bullets summarize the empirical strategy informally; this information will eventually be placed appropriately into the subsection headers below this brief summary.
    \begin{enumerate}
        \item Reject conventional TWFE-DiD estimator. This estimator essentially estimates the ATT by performing a bunch of 2 unit, 2 period DiDs, then averages the calculated treatment effects. Goodman-Bacon (2021) shows that this estimator is composed in part of DiD estimates involving \emph{two units which are both treated during both periods}. This is in violation of the spirit of DiD. These incorrect DiD estimates are known as ``forbidden comparisons.''
        \item The way to properly recover the ATT is to basically perform all of the ``non-forbidden'' comparisons possible in our panel dataset. More formally:
        \begin{enumerate}
            \item We begin with a panel dataset of properties whose Zestimates are observed monthly. At each property $i$, an eviction case concludes during month $G_i=g$, resulting in a plaintiff victory or a defendant victory. These properties are collectively known as ``cohort $g$.'' If the eviction case at property $i$ resulted in a defendant victory, let $C_i=1$
            \item Break the units into ``cohorts'' according to the month in which their cases conclude. 
            \item We estimate $ATT(g, t)$, which is the average treatment effect on the treated for cohort $g$ at time $t$, by calculating the sample analog of:

            
            $E[Y_t - Y_{g-1}|G=g] - [Y_t - Y_{g-1}|C=1]$ 

            
            \item Intuitively, $ATT(g, t)$ comes from 2 period, 2 unit difference in difference estimate. The treatment group contains all units where we observe a plaintiff victory and the case was resolved at time $g$. The control group contains all units where we observe a defendant victory and the case was resolved at time $g$. We adjust the change in observed outcomes in the treatment group by the change in observed outcomes in the control group to get our estimate.
           
        \end{enumerate}
        \item The parallel trends and no anticipation assumptions are extended to the staggered treatment context.
            \begin{enumerate}
                \item \textbf{Staggered Treatment Adoption Assumption.} Let $D_it=1$ if unit $i$ has been treated by period $t$ and $D_it=0$ otherwise. Then, for $t=1,...,\tau$, $D_{it}=1 \rightarrow D_{it+1} = 1$. Basically, once a unit becomes treated, it stays treated.
                
                \item \textbf{Parallel Trends Assumption Based On Never Treated Units}. For all $g=2,...,\tau, t=2,...,\tau$ with $t\geq g$, we have

                $E[Y_t(0)-Y_{g-1}(0)|G=g]$ = $E[Y_t(0)-Y_{g-1}(0)|C=1]$

                Basically, for every possible 2 unit, 2 period DiD that we can run in our panel dataset between treated units and never-treated units, the canonical parallel trends assumption holds.
                
                
                \item \textbf{Parallel Trends Assumption Based On Not Yet Treated Units}. For all $g=2,...,\tau, s, t=2,...,\tau$ with $t \geq g$ and $s \geq t$, we have

                $E[Y_t(0) - Y_{t-1}(0)|G=g] = E[Y_t(0) - Y_{t-1}(0)|D_s=1, G \neq g]$

                This assumption allows us to use \emph{not yet treated} observations as a comparison group for \emph{treated} units. This is clearly a stronger assumption than the previous one and I have to think hard about whether I can make it in my case. Note that the graph in section 5.3 does NOT make this assumption.

                

                
            \end{enumerate}
        \item Now, we have a bunch of $ATT(g, t)$s. How do we aggregate them?
            \begin{enumerate}
                \item Take frequency-weighted averages across $t$ to get $ATT(g)$s, or ATTs for each cohort.
                \item Take frequency-weighted averages across $g$ to get $ATT(t)$s, or ATTs for each time period.
                \item Most convincing: calculate event time $e = t - g$. Now, $ATT(g, e_i)$ expresses the ATT on cohort $g$ at time relative to the end of the eviction case involving property $i$. Take frequency weighted averages across $g$ to get $ATT(e)s$, or ATTs for each period relative to the treatment period.
            \end{enumerate}
    \end{enumerate}

    \newpage
    \begin{landscape}
       \begin{table}[H]
        \centering
        \begin{tabular}{llccccc}
\toprule
 &  & \textit{} & \multicolumn{4}{c}{\textit{Difference in Cases Won by Defendant}} \\ 
 &  & Cases Won by Plaintiff & Unweighted & \emph{p} & Weighted & \emph{p} \\
\midrule
\multirow[c]{3}{3cm}{\textit{Panel A}} & Zestimate, Jan. 2017 & 291,945.81 & 26,570.00 & 0.00 & 10,202.27 & 0.18 \\
 & Zestimate, Jan. 2018 & 320,176.82 & 27,148.19 & 0.00 & 9,364.98 & 0.26 \\
 & Change from Jan. 2018 to Jan. 2019 & 50,501.64 & 2,664.69 & 0.27 & -133.67 & 0.96 \\
\cline{1-7}
\multirow[c]{6}{3cm}{\textit{Panel B}} & Portion with bachelor's degree or higher (2010) & 0.25 & 0.01 & 0.07 & 0.01 & 0.39 \\
 & Number of jobs per sq. mile (2010) & 4,158.02 & 263.90 & 0.63 & 133.63 & 0.81 \\
 & Median household income (2016) & 52,087.32 & 2,566.82 & 0.01 & 1,178.12 & 0.25 \\
 & Population density (2010) & 9,031.64 & 944.99 & 0.01 & 483.30 & 0.18 \\
 & Portion white (2010) & 0.60 & 0.04 & 0.00 & 0.02 & 0.07 \\
 & Share with commute under 15 min. (2010) & 0.29 & -0.01 & 0.02 & -0.00 & 0.27 \\
\cline{1-7}
\textit{Panel C} & For cause & 0.12 & 0.01 & 0.68 & 0.00 & 0.82 \\
\cline{1-7}
\textit{Panel D} & Defendant is an entity & 0.01 & 0.00 & 0.37 & 0.00 & 0.67 \\
\cline{1-7}
\bottomrule
\end{tabular}

        \caption{Caption}
        \label{tab:my_label}
    \end{table} 
    \end{landscape}
    
    \subsection{Parallel Trends Assumptions in the Staggered Treatment Case}
        
        \subsection{Testing Parallel Trends Empirically Using Rambachan and Roth (2022)}
    \subsection{Aggregating Treatment Effects Across Cohort and Time}
        \subsubsection{ATTs Across Cohorts}
        \subsubsection{ATTs Across Time Periods}
        \subsubsection{Event-Study ATTs}
    
\section{Results} \label{sec:result}
    \subsection{ATTs Across Cohorts}
    \subsection{ATTs Across Time Periods}
    \subsection{Event-Study ATTs}
        \begin{figure}[H]
            \centering
            \includegraphics{output/DiD/figures/att_gt_estimates_event_study.png}
            \caption{$ATT(e)$s}
            \label{fig:my_label}
        \end{figure}


\section{Conclusion} \label{sec:conclusion}



\singlespacing
\setlength\bibsep{0pt}
\bibliographystyle{my-style}
\bibliography{Placeholder}



\clearpage

\onehalfspacing

% \section*{Tables} \label{sec:tab}
% \addcontentsline{toc}{section}{Tables}



% \clearpage

% \section*{Figures} \label{sec:fig}
% \addcontentsline{toc}{section}{Figures}




% \clearpage

% \section*{Appendix A. Placeholder} \label{sec:appendixa}
% \addcontentsline{toc}{section}{Appendix A}



\end{document}